\section{Conclusions}
\label{sec:conc}

Modern web frameworks intent to
assist developers by enabling
security features by default.
Nevertheless,
the negligent bypassing of these
measures can compromise 
the security of an application.
Furthermore,
the complex architecture of
{\sc mvc}-based applications
makes it hard to identify such hazards.
Pythia,
aims to assist developers and
security experts to identify
such issues within large code bases
and in an easy manner.
Furthermore,
our evaluation results indicate
that Pythia can be used to examine
applications in production and with
thousands of users.

Pythia is the first mechanism
that takes into account features
such as templates and their inheritance.
Notably,
the approach behind Pythia can be
applied to different contexts
such as the Laravel framework which
runs on {\sc mvc}-based applications
written in {\sc php}
(see also Section~\ref{sec:motivation}).

Future work on our mechanism
involves the creation of visualizations
based on the tool's output and the
application of further checks
when a dangerous path is identified
in a similar to
{\sc wasc}~\cite{NLC07} and
{\sc php} Aspis~\cite{PMP11}
(see Section~\ref{sec:rel}).

% \subsection*{Code Availability}
% Pythia's source code is open source and
% can be reached through the
% following {\sc url}:

% \subsection*{Acknowledgements}
% This work was supported by the
% {\sc certcoop} project,
% funded by the European Union's
% Connecting Europe Facility Telecom Call
% 2016 with Proposal Code
% 2016-{\sc el}-{\sc ia}-0123
% (Cyber Security).