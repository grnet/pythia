% MVC elements here, maybe?
\section{Background}
\label{sec:background}

The Model View Controller 
({\sc mvc}) architectural
pattern~\cite{GLH03, BD04}
is one of the most widespread design
patterns used for developing user interfaces. Specifically,
it divides an application into three
basic parts that are interconnected.
The {\it model} is the application's
data structure which is independent
of the user interface and it directly
manages the data of the application.
A {\it view} can be any output
representation of information and finally,
the {\it controller}
accepts input and converts it into
commands for the model or view.
Currently,
there are thousands of web applications
that are implemented based on the {\sc mvc}
pattern.

Our research focuses on applications
that follow an adaptation of the
{\sc mvc},
namely the Model Template
View ({\sc mtv}) pattern.
In the {\sc mtv} context
there is an object-relational mapper
({\sc orm}) that mediates between data models
and a relational database (model).
There is also a system for
processing {\sc http} requests with
a web templating system (view),
and a regular-expression-based
{\sc url} dispatcher.
Templates are parts of the
client-side code and are employed by views.

Django is a Python-based Web framework
that follows the {\sc mtv} pattern
and allows developers to build
large applications in an easy way.

Even though {\sc mvc} facilitates
structured code,
it introduces a complexity that
may lead to inconsistencies~\cite{OPM15}.
In addition,
its multiple features and tangly
structure makes it hard to manually
identify vulnerabilities such as
{\sc xss} or {\sc csrf}.
 
In particular it is difficult to trace tainted data within the 
application, because in addition to the standard components of 
the architecture, other elements (i.e. templates), which follow 
inheritance relations, are in place. 
At the same time, an automated tool can relatively easy track 
data  flows due to the formalization of coding structure that 
web frameworks have brought. 

\paragraph{MVC in Django}
Django ~\cite{django} is a Python based web framework, that is 
widely used by the community since it offers many features
for easy and quick development and because of the  
increasing growth of Python ~\cite{python_trend}.
It follows a Model Template View ({\sc mtv}) or ({\sc mvt}) 
architectural pattern ~\cite{Holovaty:2009:DGD:1572516}, 
\cite{django_mvt}, which is slightly different than the 
standard {\sc mvc}. 

Django's {\tt Template}, is a part of the client-side 
and relates to the {\tt View} in {\sc mvc} as it controls what is 
displayed to the user. The {\tt Model}, has the same role in the 
{\sc mvc} and {\sc mvc} and is part of the server-side.  
Correspondingly Django framework incorporates the {\tt View} and the 
application logic, which relates to the standard {\tt Controller}, 
since it is responsible for the communication between the 
{\tt Template} and the {\tt Model}. 



