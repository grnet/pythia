\section{Related Work}
\label{sec:rel}

Pythia is related to
various approaches that
have been developed to identify web 
application vulnerabilities
either through {\it static data-flow
analysis} or {\it dynamic taint tracking}.
We describe how each one operates,
and outline its main characteristics

Livshits et al.~\cite{LL05}
statically analyzes the {\sc cfg}
(Control Flow Graph) of a Java application
to identify variables
that carry user input and check if
they reach a sink method.
In this way they detect
possible {\sc sql} injection and
{\sc xss} defects.
{\it Pixy}~\cite{JKK06} is an open source tool that examines {\sc php} scripts
in a similar manner.
In Pixy,
rules can appear in the source of the
program in the form of annotations.
Dahse and Holz~\cite{DH14} have introduced a
refined type of data-flow analysis to
identify second-order vulnerabilities.
Note that vulnerabilities of this kind
occur when an attack payload
is first stored by the web application
on the server-side and then later on used
in a security-critical operation.

Vogt et al.~\cite{VFJKKV07} have
developed a taint tracking scheme
which operate on the server-side at runtime.
Their mechanism tracks sensitive
information at the client-side
to ensure that a script can send
sensitive user data only to the
site from which it came from.
In this way they prevent {\sc xss}
attacks that may attempt to send
the user's cookie to a malicious site.
Stock et al.~\cite{SLMS14} use
taint tracking to prevent
{\sc dom}-based {\sc xss} attacks.
Specifically, it employs a taint-enhanced JavaScript engine that
tracks the flow of malicious data.
To prevent an attack,
the scheme uses {\sc html} and JavaScript
parsers that can identify the generation of code which in turn comes from tainted data.
{\sc wasc}~\cite{NLC07} and
{\sc php} {\it Aspis}~\cite{PMP11}
apply further checks when it is
established that tainted data have
reached a sink.
In particular,
{\sc wasc} analyzes {\sc html}
responses to examine
if there is any data that contains malicious
scripts to prevent {\sc xss} attacks.
{\sc php} Aspis,
associates tainted data with metadata
such as the propagation path to
prevent {\sc sql} injection and
{\sc xss} attacks in {\sc php} applications.

% To increase automation, static analyzers are used to detect coding and/or syntax errors, examine code structure and thwart security vulnerabilities. {\sc OWASP} project has published WAP - Web Application Protection Project ~\cite{owasp_wap}, which aims to detect and even correct input validation vulnerabilities for web applications written in {\sc php}, predicting false positive results. 

There is a number of static analysis
mechanisms that are used to analyze Python
programs and applications that are
also related to Pythia.
Libraries such as
{\it Pylint}~\cite{pylint},
{\it Flake8}~\cite{py_flake},
and {\it Django Lint} ~\cite{django_lint}
are popular quality checkers that report
common programming errors such as code
smells and provide simple refactoring
suggestions.
Security wise,
the most widely used
python static analyzer is
{\it Bandit} ~\cite{bandit},
which is being actively maintained.
However,
it does not provide great insights
when analyzing large Django projects.
This is because it analyzes {\sc ast}s 
searching for dangerous code
constructs without any further analysis.
As a result,
it generates numerous false alarms.
{\it Python Taint}
({\sc pyt})~\cite{pyt} extends the
functionality of Bandit and performs
data-flow analysis thus providing more
precise results.
Nevertheless,
{\sc pyt} does not take into account
elements such as templates and
mechanisms such as inheritance.
Hence,
it cannot address vulnerabilities
like the ones described in
Section~\ref{sec:motivation}.
In addition,
it does not resolve {\sc url}s
to views to provide a detailed path.
Contrary to the above techniques,
Pythia takes into account all the
novel features of a modern,
Django-based application including
templates and decorators.
In this way it identifies vulnerabilities
that other tools fail to do so.

% Similarly Python Taint (PYT) ~\cite{pyt} is a promising static analysis tool, based on Control Flow Graph (CFG), and dataflow analysis. Although it provides adaptors for Python frameworks, it can only support analysis for Django on a superficial level, since it cannot resolve URLs to views and analyze templates, for which the runtime environment is necessary. On the contrary, as described in \ref{sec:implementation}, Pythia fosters this by design. 

% Check and improve this! 
% To summarize, this work focuses on an analyzer tailored for Django's environment, which aims to provide developers and security experts with information regarding potential sinks, leveraging on Python abstract syntax tree.  
% The inducement towards the Django framework is that although it considered to supply developers with many security mechanisms \cite{django_sec}, there are cases where either developers do not make use of these mechanisms ~\cite{nguyensecurity}, or the framework itself has vulnerabilities ~\cite{django_cve}. 
% Additionally, based on our research there is not a similar solution to {\it Pythia}, meaning that there is not an open-source tool targeted to Django based applications that exposes dangerous filters, in-view marked safe variables and decorator invocations.

