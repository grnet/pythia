\section{Evaluation}
\label{sec:eval}

We have applied Pythia on five
different open-source,
Django projects.
In four cases we found vulnerabilities.
We have reported the defects to
the developers of each project.
In all cases except for one patches were introduced,
hence we name only the applications that were fixed.
Notably,
all applications have thousands of users.

Table~\ref{tabular:apps} presents the results
of our evaluation.
Specifically,
we show the number of each vulnerability
that Pythia identified for each application,
and provide a high-level
description of the impacts.
In the following,
we provide further details
regarding the applications and the defects we found.
Pythia did not report any defects for
Misago~\cite{misago},
a forum application written in Python.

\begin{center}
\begin{table}[b]
  \caption{Evaluation Results}
  \vspace{-1.9mm}
  \begin{tabular}{|p{1.8cm}| l | lp{1.6cm}|lp{1.6cm}|}
    \hline
     \multirow{2}{*}{\textbf{Application}} &
        \multirow{2}{*}{\textbf{Version}} &
        \multicolumn{2}{c|}{\textbf{XSS}} &
        \multicolumn{2}{c|}{\textbf{CSRF}} \\ 
        & & {\#} &  {Impacts} & {\#}  & {Impacts} \\ \hline
        Anonymous Application & - & 3 & Privilege Escalation & 0 & - \\
        \hline
        Zeus & - & 1 & Privilege Escalation & 2 & Password Reset \& {\sc d}o{\sc s}  \\ \hline
        ViMa & 2.2 & 0 & - & 9 & {\sc vm} Manipulation \\ \hline
        Weblate & 3.3 & 1 & Privilege Escalation & 0 & - \\
    \hline
  \end{tabular}
  \label{tabular:apps}
\end{table}

\end{center}

\subsection{Anonymous Application}
\label{sec:okeanos}

We have analyzed an IaaS
(Infrastracture as a Service)
application with thousands of users.
In this case,
Pythia identified an {\sc xss} vulnerability
very similar to the one described in
Section~\ref{sec:motivation}.
Specifically,
one of the application's views
includes a template named
{\tt project\_application\_form.html}:

\vspace{0.8mm}
\begin{lstlisting}[language=Python, basicstyle=\footnotesize\ttfamily]
 template_name = '/projects/project_application_form.html'
\end{lstlisting}
\vspace{0.8mm}

\noindent
At some point,
the {\tt View} utilizes the template to render a form
that incorporates data stored in the database.
However,
the data are initially provided by users
through another form.
In turn,
the template above
includes another one named
{\tt form\_field.html}:

\vspace{0.8mm}
\begin{lstlisting}[language=html,basicstyle=\footnotesize\ttfamily]

    
        
    

\end{lstlisting}
\vspace{0.8mm}

\noindent
The sink occurs in {\tt form\_field.html}
where every field of a form is marked as safe
for output (line 14):

\vspace{0.8mm}
\begin{lstlisting}[language=html,basicstyle=\footnotesize\ttfamily]

<div class="form-row"
    with-errors
    with-hidden">
      {{field.errors}}
      <p class="clearfix
         required">
         {{field.label_tag}}
         {{field|safe}}
         <span class="extra-img">&nbsp;</span>
            
                <span class="info">
                    <em>more info</em>
                    <span>{{field.help_text|safe }}</span>
                </span>
            
         </p>
    </div>

\end{lstlisting}
\vspace{0.8mm}

\noindent
The above can lead to a stored {\sc xss} attack~\cite{MLPK17},
i.e. attackers can inject malicious scripts
that are first stored on the database
of the application
and then run in the browser of other users
(including administrators),
thus stealing their cookies.

% a template was used to render any form field marking the field itself as safe for output. This template was being included in various places. In some of them, user data were being passed as form fields, leading to Stored XSS in certain places. This vulnerability could be used to escalate privileges from user to admin. It was relatively easy to review Pythia's output and figure out which occurrences of the aforementioned vulnerable template were indeed real vulnerabilities since we only had to evaluate the data sources.

\subsection{Zeus}
\label{sec:zeus}

{\it Zeus}~\cite{zeus-jets, zeus}
is an e-voting application
that has been in production
since late 2012 and
has been used in more than 500
real-world elections involving
more than 55,000 voters~\cite{pnrmx-del}.

The {\sc xss} vulnerability
that was discovered in Zeus is very
similar to the one described earlier.
An interesting observation
was that the view that included
the template with the sink,
was not accessible through the
application's
{\sc gui} (Graphical User Interface).
However,
after examining the code of the application,
attackers could reach the view through
a specific {\sc url}.
Hence,
a vulnerability scanner that examines
a running instance of the application
would not discover the vulnerability.
Contrary to such scanners,
Pythia's
ability to map {\sc url}s to views
(as described in
Subsection~\ref{sec:implementation})
led to the identification of the
dangerous path.

Apart from the {\sc xss} defect,
Pythia identified two
to {\sc csrf} vulnerabilities:
one that can be employed to reset
passwords and another that
can be a denial of service attack vector.

Specifically,
in the application version that we examined,
the password reset functionality was
accessed through a {\sc get} request from
a specific {\sc url} which incorporated
the {\sc id} (i.e. a number) of a user
as a parameter.
% \url{https://zeus-testing.grnet.gr/zeus/account_administration/reset_password_ confirmed/?uid=<user-id>}
A valid attack scenario in this case,
would be to host a website that sends
{\sc ajax} requests to the application
to reset every password.
Given that the users' {\sc id}s
are stored in a sequential way
in the database this is easy to accomplish.
This would cause organizational mess
during an election since the administrator
or anyone else would not be able to log-in.

Missing {\sc csrf} protection on
another {\sc post} form could lead to
a Denial of Service ({\sc d}o{\sc s})
attack through a {\sc csrf} attack.
In particular,
attackers could host a website,
lure victims to it,
and submit the form multiple times
through the victim's browser.
The above would lead to resource drain
due to the required computing resources
needed by the attacked endpoint.

\subsection{ViMa}
\label{sec:vima}

{\it ViMa} (Virtual Machines)~\cite{vima},
provides to the Greek academic community
access to shared computing and network resources that can be used
for production purposes.
Currently,
it is used by a hundreds of researchers
and practitioners.

Pythia identified numerous
{\tt @csrf\_exempt} decorators
that can lead to 9 types of
{\sc csrf} attacks.
Through these attacks a malicious
user can perform different actions including:
rebooting a virtual machine of another user,
shutting it down and destroying it. This attack
requires that the victim is lured to browse to a malicious
website that performs these actions using the
victim's credentials, leveraging the lack of
{\sc csrf} protection.
For instance,
the following decorator
allows an attacker to reboot
a virtual machine of another user:

\vspace{0.8mm}
\begin{lstlisting}[language=Python, basicstyle=\footnotesize\ttfamily]
@login_required
@csrf_exempt
@check_instance_auth
@check_admin_lock
def reboot(request, cluster_slug, instance):
    cluster = get_object_or_404(Cluster, slug=cluster_slug)
    [...]
\end{lstlisting}
\vspace{0.8mm}

\subsection{Weblate}
\label{sec:weblate}
\begin{comment}
https://github.com/WeblateOrg/weblate/blob/d84dfdbdfaba0d31588fd3bc9253e6d75d03e644/weblate/trans/views/basic.py#L91-L129
\end{comment}
{\it Weblate}~\cite{weblate} is a web-based
translation management system.
It is  used for translating many free
software project as well as commercial ones,
including
{\it phpMyAdmin}~\cite{phpma} and
{\it Debian Handbook}~\cite{dhbk}.

In the case of Weblate,
user data stored in the database
were being marked as safe
even before being sent to the view's corresponding template (in-view sink).
In particular,
the name of a project is provided by a user.
Then,
this information is marked as safe by
the application and is included on
the output. {\footnote{\url{https://github.com/WeblateOrg/weblate/blob/d84dfdbdfaba0d31588fd3bc9253e6d75d03e644/weblate/trans/views/basic.py\#L91-L129}}}
% TODO: add code.
% From the template's perspective,
% Django's default behaviour of sanitizing
% output was not deactivated.
This issue led to stored {\sc xss} attacks
that could be used to elevate any user's privileges to the ones of an administrator.
% TODO: point to the corresponding patch.

\subsection{False Positives and Negatives}
\label{sc:fa}

A potentially dangerous path
may involve a source that will
not include user data
e.g. a variable that is instantiated
by retrieving data from the database
and in turn the data are auto-generated
by the application.
Pythia will report the path above
producing a false alarm.
An alarm like this was produced
when we examined the
``Anonymous Application".
Specifically,
the path involved the terms and
conditions of the
``Anonymous Application"
which are marked as safe by
the template when they are
displayed to the user.
However,
such information is not useless.
After seeing such a path,
a developer can note it down and
be careful in the future regarding the
identified sink.

In general,
false positives in related mechanisms
(either data-flow analysis or
taint tracking techniques)
is a known issue~\cite{ME07, MLPK17}.
Pythia though should be seen
as an assistant for developers
who want to have a thorough
security perspective
of their application,
and not as a vulnerability detector.

A false negative may appear in the following
way: consider a {\tt mark\_safe} construct
that exists in a function invoked by a view,
Pythia will not report this case because
it does not examine the code of the invoked functions.
However, we can perform a more in-depth analysis and
include such cases too.
