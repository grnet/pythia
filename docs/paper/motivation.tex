
\section{Vulnerability Inheritance in Django-based Applications}
\label{sec:motivation}
\vspace{-0.5mm}

Development frameworks enforce security mechanisms such as the 
sanitization of {\sc html} output by default.
In this way, certain characters
(e.g. '{\tt <}' and '{\tt \&}')
are not interpreted as special
control characters.
However,
the frameworks also
allow programmers to
bypass such mechanisms to support
specific use cases which 
in turn, can lead to serious
vulnerabilities including {\sc xss} and {\sc csrf}.

A serious issue emerges when this bypassing
happens in {\it templates}, 
a feature that provides a way to generate {\sc html} dynamically.
Templates are supported by several frameworks
including Django and Laravel.
In general,
templates contain static parts of the 
desired {\sc html} output,
together with some special 
syntax describing how the dynamic
content will be inserted.

When developers insert
{\sc html} content into a template,
they need to mark it as ``safe", 
so that the framework itself
treats it as such, 
and avoid escaping {\sc html} characters.
In this case developers have to be sure about the
origins of the content
because if the content 
includes user input,
attackers can perform an 
{\sc xss} attack.

To track such a vulnerability is not trivial
because templates can either
{\it include} or {\it inherit}
other {\tt templates}.
Hence,
it is difficult to find which
views are affected
by a potentially vulnerable template.
Figure~\ref{fig:defect},
highlights this issue,
i.e. a template that marks content
as safe is included by multiple other
ones within the application.
In turn, 
these templates are used by different 
views that process content either from user 
input, or loaded from the model. 

Another issue occurs when
developers use the 
{\tt @csrf\_exempt} decorator to disable 
{\sc csrf} protection in Django
views ~\cite{csrf_exempt}.
Specifically,
Django allows developers to
inject a {\sc csrf} token 
for every form rendered and then
expect the client 
to supply this token with the
corresponding {\sc post}  request.
This mechanism is disabled when the {\tt @csrf\_exempt} 
feature is used.

To identify the aforementioned issues
during a code review,
security experts have to find all
the occurrences 
of the different features and
track which views are affected manually.
Such a task could even be performed
using {\tt grep} command.
However this would require an
exhaustive  and repetitive search
from the security expert's side.

Additionally, 
tracking the various intermediate entities that inherit 
such properties is not an easy task in modern,
large applications. 
Thus,
some level of automation is
needed to cope with 
large code-bases.
Our approach provides a way to
automatically discover 
data flows when such features
are employed and 
warn security engineers
about potentially vulnerable paths.

\vspace{-2mm}
